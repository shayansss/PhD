\Ac{ac}, a soft tissue essential for pain-free movement, faces significant biomechanical stress, making it a prime subject for research, especially via non-invasive computer simulations. These simulations often employ time-consuming numerical methods like \ac{fe} analysis. \Ac{ai}, offering a faster alternative, utilizes \ac{ml} as a surrogate model, which replicates numerical analysis using samples of numerical data. However, this approach can be inefficient due to the requirement for extensive training and large datasets. This study introduces an efficient, advanced biomechanical model using multi-physics modeling and a novel \ac{psa}. It generates a limited number of training samples across various fidelities and scales. Subsequently, we propose efficient, multi-fidelity, generalizable surrogate models with innovative preprocessing and training algorithms. These models integrate non-intrusive reduced-order modeling of the multi-physics equation with upstream \ac{ml}. Our empirical results underline the importance of pre-stressing in multiphasic modeling, which entails time-intensive numerical execution. This highlights the utility of efficient surrogate modeling. The surrogate models are then empirically tested, showing a substantial reduction in computational time. Moreover, the scalability and generalizability of the \ac{hml} framework are rigorously evaluated through multiscale simulations. The findings demonstrate the significant impact of our proposed physics-constrained \ac{da} and \ac{gnn} implementation. Therefore, while this study potentially advances the field with the developed \ac{psa} algorithm, it also presents an effective approach to address computational challenges and data scarcity in \ac{ml}-based surrogate modeling of cartilage biomechanics. Besides, this research, though tailored to specific cartilage modeling, potentially paves the way for broader applications of our synergistic methodology through its open-source availability.
