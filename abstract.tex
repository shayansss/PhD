\Ac{ac}, a crucial soft tissue for pain-free movement, undergoes significant biomechanical stress, warranting research through non-invasive computer simulations. Such simulations commonly use time-intensive numerical methods, such as \ac{fe} analysis. \Ac{ai}, with \ac{ml} as a surrogate model, presents a faster alternative, imitating numerical analysis with numerical data samples. Despite its potential, this method often requires extensive training and large datasets, which is inefficient. This study first introduces an efficient biomechanical model incorporating multi-physics modeling and a novel \ac{psa}, generating few training samples across different fidelities and scales. We then propose efficient, multi-fidelity, generalizable surrogate models, enhanced by innovative preprocessing and training algorithms. Finally, the research code, including our developed Fortran subroutines and Python scripts, is open-sourced. Empirical results emphasize the significance of pre-stressing in multiphasic modeling and the value of efficient surrogate modeling in reducing computational time. The scalability and generalizability of the \ac{hml} framework are confirmed through multiscale simulations, demonstrating the impact of our physics-constrained \ac{da} and \ac{gnn} implementation. Therefore, while this study potentially advances
the field with the developed \ac{psa} implementation, it also presents a straightforward approach to address computational challenges and data scarcity in ML-based surrogate modeling of cartilage biomechanics. Such research, though tailored to specific cartilage modeling, potentially paves the way for broader applications of our synergistic methodology through its open-source availability.







